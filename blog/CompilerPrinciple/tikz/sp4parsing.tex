\documentclass[border=10pt,margin=5pt,tikz,dvisvgm,rgb,utf8]{standalone}
\usepackage{ctex,xeCJK}  % 中文环境
\setCJKmainfont[BoldFont=Source Han Sans SC]{Source Han Serif SC}
\usepackage{calc,fontawesome,forest,smartdiagram,xcolor}
\usetikzlibrary{animations,arrows,automata,graphs,matrix,positioning,shadows,shapes}

\begin{document}
\renewcommand{\baselinestretch}{0.4}

\begin{tikzpicture}
  \node[text width=-3em](start){\tiny};
  \node[align=center, draw=black, right=2em of start](lexical_analyzer){\tiny{词法}\\ \tiny{分析器}};
  \node[align=center, draw=black, right=3.6em of lexical_analyzer](parsing){\tiny{语法}\\ \tiny{分析器}};
  \node[align=center, draw=black, right=3.2em of parsing](other){\tiny{其他}\\ \tiny{前端部分}};
  \node[text width=-3em, right=2.5em of other](end){\tiny};
  \node[align=center, draw=black, below=1em of parsing](symbol_table){\tiny 符号表};

  \path[->]
  (start) edge node[above=-2pt]{\tiny 源程序} (lexical_analyzer)
  (other) edge node[above=-2pt]{\tiny 中间表示} (end);
  \path[dashed, ->]
  (parsing) edge node[above=-2pt]{\tiny 语法分析树} (other);

  \draw (lexical_analyzer)
  edge[bend left=20, draw=none] coordinate[at start](la2p_a) coordinate[at end](la2p_b) (parsing)
  edge[bend right=20, draw=none] coordinate[at start](p2la_b) coordinate[at end](p2la_a) (parsing);
  \path[->]
  (la2p_a) edge node[above=-2pt]{\tiny 词法单元} (la2p_b)
  (p2la_a) edge node[below=-2pt]{\tiny 获取词法单元} (p2la_b);


  \draw (symbol_table)
  edge[out=180, in=270, draw=none] coordinate[at start](st_start1) coordinate[at end](la_end) (lexical_analyzer)
  edge[out=0, in=270, draw=none] coordinate[at start](st_start2) coordinate[at end](other_end) (other);
  \path[<->]
  (la_end) edge (st_start1)
  (parsing) edge (symbol_table)
  (other_end) edge (st_start2);
\end{tikzpicture}

\end{document}
