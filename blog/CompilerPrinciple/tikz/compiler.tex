\documentclass[border=10pt,margin=5pt,tikz,dvisvgm,rgb]{standalone}
\usepackage{ctex,xeCJK}  % 中文环境
\setCJKmainfont[BoldFont=Source Han Sans SC]{Source Han Serif SC}
\usepackage{calc,fontawesome,forest,smartdiagram,xcolor}
\usetikzlibrary{animations,arrows,automata,graphs,matrix,positioning,shadows,shapes}

\begin{document}
\renewcommand{\baselinestretch}{0.4}

\begin{tikzpicture}
  % 编译器
  \node[align=center, text width=2.7em, minimum height=1em](source1){\small 源程序};
  \node[align=center, draw=black, text width=2.7em, below=of source1](compiler1){\small 编译器};
  \node[align=center, text width=3.6em, minimum height=1em, below=of compiler1](object1){\small 目标程序};
  \draw[->](source1) -- (compiler1);
  \draw[->](compiler1) -- (object1);

  % 解释器
  \node[align=center, text width=0.9em, minimum height=2.7em, right=5em of source1](source2){\small 源程序};
  \node[align=center, text width=0.9em, minimum height=2.7em, right=1em of source2](input2){\small 输入};
  \node[align=center, draw=black, text width=2.7em, below=of $(source2)!0.5!(input2)$](interpreter2){\small 解释器};
  \node[align=center, text width=1.8em, below=of interpreter2](output2){\small 输出};
  \draw[->](source2) -- (interpreter2);
  \draw[->](input2) -- (interpreter2);
  \draw[->](interpreter2) -- (output2);

  % 混合编译器
  \node[align=center, text width=2.7em, right=5em of input2](source3){\small 源程序};
  \node[align=center, draw=black, text width=2.7em, below=1.5em of source3](transducer3){\small 翻译器};
  \node[align=center, text width=3.6em, below=1.5em of transducer3](mid3){\small 中间程序};
  \node[align=center, text width=1.8em, below=0.5em of mid3](input3){\small 输入};
  \node[align=center, text width=1.8em, right=1.8em of transducer3](output3){\small 输出};
  \node[align=center, draw=black, text width=2.7em, below=of output3](vm3){\small 虚拟机};
  \draw[->](source3) -- (transducer3);
  \draw[->](transducer3) -- (mid3);
  \draw[->](mid3) -- (vm3);
  \draw[->](input3) -- (vm3);
  \draw[->](vm3) -- (output3);
\end{tikzpicture}

\end{document}
